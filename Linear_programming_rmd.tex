\documentclass[]{article}
\usepackage{lmodern}
\usepackage{amssymb,amsmath}
\usepackage{ifxetex,ifluatex}
\usepackage{fixltx2e} % provides \textsubscript
\ifnum 0\ifxetex 1\fi\ifluatex 1\fi=0 % if pdftex
  \usepackage[T1]{fontenc}
  \usepackage[utf8]{inputenc}
\else % if luatex or xelatex
  \ifxetex
    \usepackage{mathspec}
  \else
    \usepackage{fontspec}
  \fi
  \defaultfontfeatures{Ligatures=TeX,Scale=MatchLowercase}
\fi
% use upquote if available, for straight quotes in verbatim environments
\IfFileExists{upquote.sty}{\usepackage{upquote}}{}
% use microtype if available
\IfFileExists{microtype.sty}{%
\usepackage{microtype}
\UseMicrotypeSet[protrusion]{basicmath} % disable protrusion for tt fonts
}{}
\usepackage[margin=1in]{geometry}
\usepackage{hyperref}
\hypersetup{unicode=true,
            pdftitle={Understanding Linear Programming in R},
            pdfauthor={Vicky Crasto},
            pdfborder={0 0 0},
            breaklinks=true}
\urlstyle{same}  % don't use monospace font for urls
\usepackage{color}
\usepackage{fancyvrb}
\newcommand{\VerbBar}{|}
\newcommand{\VERB}{\Verb[commandchars=\\\{\}]}
\DefineVerbatimEnvironment{Highlighting}{Verbatim}{commandchars=\\\{\}}
% Add ',fontsize=\small' for more characters per line
\usepackage{framed}
\definecolor{shadecolor}{RGB}{248,248,248}
\newenvironment{Shaded}{\begin{snugshade}}{\end{snugshade}}
\newcommand{\KeywordTok}[1]{\textcolor[rgb]{0.13,0.29,0.53}{\textbf{{#1}}}}
\newcommand{\DataTypeTok}[1]{\textcolor[rgb]{0.13,0.29,0.53}{{#1}}}
\newcommand{\DecValTok}[1]{\textcolor[rgb]{0.00,0.00,0.81}{{#1}}}
\newcommand{\BaseNTok}[1]{\textcolor[rgb]{0.00,0.00,0.81}{{#1}}}
\newcommand{\FloatTok}[1]{\textcolor[rgb]{0.00,0.00,0.81}{{#1}}}
\newcommand{\ConstantTok}[1]{\textcolor[rgb]{0.00,0.00,0.00}{{#1}}}
\newcommand{\CharTok}[1]{\textcolor[rgb]{0.31,0.60,0.02}{{#1}}}
\newcommand{\SpecialCharTok}[1]{\textcolor[rgb]{0.00,0.00,0.00}{{#1}}}
\newcommand{\StringTok}[1]{\textcolor[rgb]{0.31,0.60,0.02}{{#1}}}
\newcommand{\VerbatimStringTok}[1]{\textcolor[rgb]{0.31,0.60,0.02}{{#1}}}
\newcommand{\SpecialStringTok}[1]{\textcolor[rgb]{0.31,0.60,0.02}{{#1}}}
\newcommand{\ImportTok}[1]{{#1}}
\newcommand{\CommentTok}[1]{\textcolor[rgb]{0.56,0.35,0.01}{\textit{{#1}}}}
\newcommand{\DocumentationTok}[1]{\textcolor[rgb]{0.56,0.35,0.01}{\textbf{\textit{{#1}}}}}
\newcommand{\AnnotationTok}[1]{\textcolor[rgb]{0.56,0.35,0.01}{\textbf{\textit{{#1}}}}}
\newcommand{\CommentVarTok}[1]{\textcolor[rgb]{0.56,0.35,0.01}{\textbf{\textit{{#1}}}}}
\newcommand{\OtherTok}[1]{\textcolor[rgb]{0.56,0.35,0.01}{{#1}}}
\newcommand{\FunctionTok}[1]{\textcolor[rgb]{0.00,0.00,0.00}{{#1}}}
\newcommand{\VariableTok}[1]{\textcolor[rgb]{0.00,0.00,0.00}{{#1}}}
\newcommand{\ControlFlowTok}[1]{\textcolor[rgb]{0.13,0.29,0.53}{\textbf{{#1}}}}
\newcommand{\OperatorTok}[1]{\textcolor[rgb]{0.81,0.36,0.00}{\textbf{{#1}}}}
\newcommand{\BuiltInTok}[1]{{#1}}
\newcommand{\ExtensionTok}[1]{{#1}}
\newcommand{\PreprocessorTok}[1]{\textcolor[rgb]{0.56,0.35,0.01}{\textit{{#1}}}}
\newcommand{\AttributeTok}[1]{\textcolor[rgb]{0.77,0.63,0.00}{{#1}}}
\newcommand{\RegionMarkerTok}[1]{{#1}}
\newcommand{\InformationTok}[1]{\textcolor[rgb]{0.56,0.35,0.01}{\textbf{\textit{{#1}}}}}
\newcommand{\WarningTok}[1]{\textcolor[rgb]{0.56,0.35,0.01}{\textbf{\textit{{#1}}}}}
\newcommand{\AlertTok}[1]{\textcolor[rgb]{0.94,0.16,0.16}{{#1}}}
\newcommand{\ErrorTok}[1]{\textcolor[rgb]{0.64,0.00,0.00}{\textbf{{#1}}}}
\newcommand{\NormalTok}[1]{{#1}}
\usepackage{longtable,booktabs}
\usepackage{graphicx,grffile}
\makeatletter
\def\maxwidth{\ifdim\Gin@nat@width>\linewidth\linewidth\else\Gin@nat@width\fi}
\def\maxheight{\ifdim\Gin@nat@height>\textheight\textheight\else\Gin@nat@height\fi}
\makeatother
% Scale images if necessary, so that they will not overflow the page
% margins by default, and it is still possible to overwrite the defaults
% using explicit options in \includegraphics[width, height, ...]{}
\setkeys{Gin}{width=\maxwidth,height=\maxheight,keepaspectratio}
\IfFileExists{parskip.sty}{%
\usepackage{parskip}
}{% else
\setlength{\parindent}{0pt}
\setlength{\parskip}{6pt plus 2pt minus 1pt}
}
\setlength{\emergencystretch}{3em}  % prevent overfull lines
\providecommand{\tightlist}{%
  \setlength{\itemsep}{0pt}\setlength{\parskip}{0pt}}
\setcounter{secnumdepth}{0}
% Redefines (sub)paragraphs to behave more like sections
\ifx\paragraph\undefined\else
\let\oldparagraph\paragraph
\renewcommand{\paragraph}[1]{\oldparagraph{#1}\mbox{}}
\fi
\ifx\subparagraph\undefined\else
\let\oldsubparagraph\subparagraph
\renewcommand{\subparagraph}[1]{\oldsubparagraph{#1}\mbox{}}
\fi

%%% Use protect on footnotes to avoid problems with footnotes in titles
\let\rmarkdownfootnote\footnote%
\def\footnote{\protect\rmarkdownfootnote}

%%% Change title format to be more compact
\usepackage{titling}

% Create subtitle command for use in maketitle
\newcommand{\subtitle}[1]{
  \posttitle{
    \begin{center}\large#1\end{center}
    }
}

\setlength{\droptitle}{-2em}
  \title{Understanding Linear Programming in R}
  \pretitle{\vspace{\droptitle}\centering\huge}
  \posttitle{\par}
  \author{Vicky Crasto}
  \preauthor{\centering\large\emph}
  \postauthor{\par}
  \predate{\centering\large\emph}
  \postdate{\par}
  \date{March 14, 2018}


\begin{document}
\maketitle

\section{Basic concepts in Linear
Programming}\label{basic-concepts-in-linear-programming}

\subsubsection{Definition}\label{definition}

Linear programming is a simple technique where we depict complex
relationships through linear functions and then find the optimum points.
Linear Programming is used for solving optimization (Maximise/Minimise)
problem in analytics.

\emph{Linear Programming can be of two types}

\begin{itemize}
\tightlist
\item
  Integers Linear Programming
\item
  Mixed Integers Linear Programming (Branch \& Bound Method, Cutting
  Plane Method etc.)
\end{itemize}

\subsubsection{Structure of a Linear Programming
Model:}\label{structure-of-a-linear-programming-model}

A formulation of a linear program in its canonical form of maximum is:

Max z = c\textsubscript{1}x\textsubscript{1} +
c\textsubscript{2}x\textsubscript{2} + . +
c\textsubscript{n}x\textsubscript{n}

Subject to:

a\textsubscript{11}x\textsubscript{1} +
a\textsubscript{12}x\textsubscript{2} + . +
a\textsubscript{1n}x\textsubscript{n} \textless{}= b\textsubscript{1}

a\textsubscript{21}x\textsubscript{1} +
a\textsubscript{22}x\textsubscript{2} + . +
a\textsubscript{2n}x\textsubscript{n} \textless{}= b\textsubscript{2}

.

a\textsubscript{m1}x\textsubscript{1} +
a\textsubscript{m2}x\textsubscript{2} + . +
a\textsubscript{mn}x\textsubscript{n} \textless{}= b\textsubscript{m}

x\textsubscript{i} \textgreater{}= 0

\paragraph{Objective Function}\label{objective-function}

It is defined as the objective of making decisions (Maximise or
Minimise)

\paragraph{Decision Variables}\label{decision-variables}

It is the variables which decides the output. They are affected by the
cost coefficients (cj).

\paragraph{Constraints}\label{constraints}

It restricts or limits the value of the decision variables. A set of m
constraints, in which a linear combination of the variables affected by
coefficients aij has to be less or equal than its right-hand side value
bi (constraints with signs greater or equal or equalities are also
possible)

\paragraph{Non-negative restrictions}\label{non-negative-restrictions}

Decision variables should have bounds. (Non-negative)

\emph{So, Linear programming problem can be alternatively defined as
follows:}

\begin{itemize}
\item
  It consists of optimising (minimize or maximize) the value of a linear
  objective function of a vector of decision variables, considering that
  the variables can only take the values defined by a set of linear
  constraints.
\item
  Linear programming is a case of mathematical programming, where
  objective function and constraints are linear.
\item
  Constraints of Linear Programming defines a feasible region. No. of
  variables determines the shape of feasible region. For a Linear
  Programming of n variables, the shape of the feasible region would be
  n dimensional convex polytope. (A convex polytope is a special case of
  a polytope, having the additional property that it is also a convex
  set of points in the n-dimensional space Rn.
\end{itemize}

\subsubsection{Steps of Problem
Formulation}\label{steps-of-problem-formulation}

\begin{itemize}
\tightlist
\item
  Identifying the decision variables
\item
  Writing the objective function
\item
  Writing the constraints
\item
  Writing the non-negativity restrictions
\end{itemize}

\subsubsection{Duality in Linear
Programming:}\label{duality-in-linear-programming}

It states that every LPP has another LPP related to it and so can be
derived from it. Original LPP is called ``Primal'' \& the derived LPP is
called ``Dual.''

Let's consider a MAX LPP in its canonical form:

\textbf{Primal}

MAX z = c'x

s.t. Ax \textless{}= b

x \textgreater{}= 0

\textbf{Dual}

MIN w = u'b

s.t. u'A \textgreater{}= c'

u \textgreater{}= 0

\paragraph{Important Points}\label{important-points}

\begin{itemize}
\tightlist
\item
  Each variable of dual is linked with a constraint of primal
\item
  Each constraint of dual is linked with a variable of the primal
\end{itemize}

\paragraph{Properties of Primal \& Dual
Relationship:}\label{properties-of-primal-dual-relationship}

\begin{itemize}
\tightlist
\item
  Dual of dual is a primal
\item
  If a linear program has a bounded optimum, its primal has also a
  bounded optimum and both have the same value
\end{itemize}

\section{Solving LP problem using Simplex
Method}\label{solving-lp-problem-using-simplex-method}

\paragraph{Algebric Method - Simplex
Method}\label{algebric-method---simplex-method}

\textbf{Step 1:} Add Slack Variable or subtract non-negative Surplus
Variable to the constraints to convert it into equality constraint.

Let us take a problem and illustrate the steps:

Maximise

Z = 6X\textsubscript{1}+ 5X\textsubscript{2}

Subject to

X\textsubscript{1} + X\textsubscript{2} \textless{}= 5

3X\textsubscript{1} + 2X\textsubscript{2} \textless{}= 12

X\textsubscript{1}, X\textsubscript{2} \textgreater{}= 0

Maximise

Z = 6X\textsubscript{1}+ 5X\textsubscript{2} + 0X\textsubscript{3} +
0X\textsubscript{4}

X\textsubscript{1} + X\textsubscript{2} + X\textsubscript{3} = 5 .. (eq
1)

3X\textsubscript{1} + 2X\textsubscript{2} + X\textsubscript{4} = 12 ..
(eq 2)

X\textsubscript{1}, X\textsubscript{2}, X\textsubscript{3},
X\textsubscript{4} \textgreater{}= 0

\textbf{Step 2:} An initial basic feasible solution is derived by
substituting 0 for X\textsubscript{1} \& X\textsubscript{2} in the eq 1
\& eq 2. Also, express the basic feasible variables by non-basic
variables

\emph{Iteration 1:}

X\textsubscript{3} = 5 - X\textsubscript{1} - X\textsubscript{2}

X\textsubscript{4} = 12 - 3X\textsubscript{1} - 2X\textsubscript{2}

Z = 6X\textsubscript{1}+ 5X\textsubscript{2}

\textbf{Step 3:} Currently, Z = 0. As both the co-efficient of
X\textsubscript{1} \& X\textsubscript{2} are positive, we will increase
any one of the variables X\textsubscript{1} \& X\textsubscript{2} to
increase Z. Here, we will increase the variable X\textsubscript{1} as
the rate of change of Z is more with variable X\textsubscript{1}. We can
increase the variable X\textsubscript{1} maximum to 4. (As, increasing
it beyond 4 will make the variable X\textsubscript{4} negative)

\emph{Iteration 2:}

3X\textsubscript{1} = 12 - 2X\textsubscript{2} - X\textsubscript{4} ..
from (eq 2)

X\textsubscript{1} = 4 - 2/3 X\textsubscript{2} - 1/3 X\textsubscript{4}

X3 = 1 - 1/3 X\textsubscript{2} - 1/3 X\textsubscript{4} \ldots{} from
(eq 1)

Z = 24 + X\textsubscript{2} - 2X\textsubscript{4}

From here, we get another \textbf{Basic Feasible Solution:}
X\textsubscript{1} = 4, X\textsubscript{3} = 1, X\textsubscript{2},
X\textsubscript{4} = 0

So, next X\textsubscript{2} will be increased to increase Z.
X\textsubscript{2} can be increased upto 3.

\emph{Iteration 3:}

1/3 X\textsubscript{2} = 1 - X\textsubscript{3} + 1/3 X\textsubscript{4}

X\textsubscript{2} = 3 - 3X\textsubscript{3} + X\textsubscript{4}

X\textsubscript{1} = 2 + 2X\textsubscript{3} - X\textsubscript{4}

Z = 24 + (3 - 3X\textsubscript{3} + X\textsubscript{4}) -
2X\textsubscript{4}

Z = 27 - 3X\textsubscript{3} - X\textsubscript{4}

X\textsubscript{1} = 2, X\textsubscript{2} = 3 \& Z = 27

\paragraph{Tabular Form of Simplex
Method:}\label{tabular-form-of-simplex-method}

\textbf{Maximise}

Z = 6X\textsubscript{1}+ 5X\textsubscript{2} + 0X\textsubscript{3} +
0X\textsubscript{4}

X\textsubscript{1} + X\textsubscript{2} + X\textsubscript{3} = 5 .. (eq
1)

3X\textsubscript{1} + 2X\textsubscript{2} + X\textsubscript{4} = 12 ..
(eq 2)

X\textsubscript{1}, X\textsubscript{2}, X\textsubscript{3},
X\textsubscript{4} \textgreater{}= 0

\textbf{Step 1:} Co-efficient of the variable in the objective function
is written on the top row. Co-efficient of the variables on the
constraints are written on the corresponding rows. RHS of the
constraints are written the RHS Column of the corresponding Row.

\textbf{Step 2:} C\textsubscript{j} - Co-efficient of the variable.
Z\textsubscript{j} - Sum product of 1st Column \& the Corresponding
Variable Column. So, C\textsubscript{j} - Z\textsubscript{j} will be as
follows

e.g. 6 - (0\emph{1+0}3) = 6, 5 - (0\emph{1+0}2) = 5

\begin{longtable}[]{@{}ccccccc@{}}
\toprule
& X\textsubscript{1} = 6 & X\textsubscript{2} = 5 & X\textsubscript{3} =
0 & X\textsubscript{4} = 0 & RHS & Theta\tabularnewline
\midrule
\endhead
X\textsubscript{3} (0) & 1 & 1 & 1 & 0 & 5 & 5\tabularnewline
X\textsubscript{4} (0) & 3 & 2 & 0 & 1 & 12 & 4 (Pivot
Row)\tabularnewline
C\textsubscript{j} - Z\textsubscript{j} & 6 & 5 & 0 & 0 & 0 &
0\tabularnewline
X\textsubscript{3} (0) & 0 & 1/3 & 1 & -1/3 & 1 & 3 (Pivot
Row)\tabularnewline
X\textsubscript{1} (6) & 1 & 2/3 & 0 & 1/3 & 4 & 6\tabularnewline
C\textsubscript{j} - Z\textsubscript{j} & 0 & 1 & 0 & -2 & 24
&\tabularnewline
X\textsubscript{2} (5) & 0 & 1 & 3 & -1 & 3 &\tabularnewline
X\textsubscript{1} (6) & 1 & 0 & -2 & 1 & 2 &\tabularnewline
C\textsubscript{j} - Z\textsubscript{j} & 0 & 0 & -3 & -1 & 27
&\tabularnewline
\bottomrule
\end{longtable}

\textbf{Step 3:} Select the variable with the highest C\textsubscript{j}
- Z\textsubscript{j} value. Here it is 6, variable X\textsubscript{1}.

Now, we will calculate the (Theta) by dividing RHS with the
corresponding co-efficient of X\textsubscript{1}.

Now, we will select the minimum value of the Theta as the limiting
value. Corresponding row will become the Pivot Row \& the Co-efficient
of X\textsubscript{1} in the Pivot Row will be the Pivot Element (3).

In the next step the Pivot Row will be replaced by X\textsubscript{1}.

\textbf{Step 4:} Row operations should be performed in such a way that
the variables under consideration is equal to the identity matrix.

For X\textsubscript{3}, (X\textsubscript{3} -1/3X\textsubscript{4}) \&
for X\textsubscript{1}, X\textsubscript{4}/3. Again, we calculate the
C\textsubscript{j} - Z\textsubscript{j}, and find out the pivot row,
pivot element \& the limiting value.

\textbf{Step 5:} After calculating the 3rd C\textsubscript{j} -
Z\textsubscript{j}, there is no way to maximise the RHS, without
violating the non-negativity condition.

So, the final answer will be X\textsubscript{1} = 3, X\textsubscript{2}
= 2 \& Z = 27.

\section{Solving linear programming problem in R using
lpsolve}\label{solving-linear-programming-problem-in-r-using-lpsolve}

A car company produces 2 models, model A and model B. Long-term
projections indicate an expected demand of at least 100 model A cars and
80 model B cars each day. Because of limitations on production capacity,
no more than 200 model A cars and 170 model B cars can be made daily. To
satisfy a shipping contract, a total of at least 200 cars much be
shipped each day. If each model A car sold results in a \$2000 loss, but
each model B car produces a \$5000 profit, how many of each type should
be made daily to maximize net profits?

\subsubsection{Formulating the problem}\label{formulating-the-problem}

Let the A and B be the number of cars of the two models to be produced.

\textbf{Objective function} : Maximize ( -2000A + 5000B)

\textbf{Constraints} :

\emph{Demand Constraint} A\textgreater{}= 100 , B \textgreater{}=80

\emph{Production Constraint} A\textless{}= 200 , B\textless{}=170

\emph{Shipping Constraint} A+B \textgreater{}=200

\paragraph{Using lpSolveAPI to solve this problem in
R}\label{using-lpsolveapi-to-solve-this-problem-in-r}

\textbf{Install package the IpSolve and IpSovleAPI}

\begin{Shaded}
\begin{Highlighting}[]
\KeywordTok{library}\NormalTok{(lpSolve)}
\end{Highlighting}
\end{Shaded}

\begin{verbatim}
## Warning: package 'lpSolve' was built under R version 3.3.2
\end{verbatim}

\begin{Shaded}
\begin{Highlighting}[]
\KeywordTok{library}\NormalTok{(lpSolveAPI)}
\end{Highlighting}
\end{Shaded}

\begin{verbatim}
## Warning: package 'lpSolveAPI' was built under R version 3.3.2
\end{verbatim}

\textbf{Creating the model}

\begin{Shaded}
\begin{Highlighting}[]
\NormalTok{model <-}\StringTok{ }\KeywordTok{make.lp}\NormalTok{(}\DataTypeTok{ncol=}\DecValTok{2}\NormalTok{)}

\NormalTok{m1 <-}\KeywordTok{lp.control}\NormalTok{(model,}\DataTypeTok{sense=}\StringTok{"max"}\NormalTok{, }\DataTypeTok{verbose=}\StringTok{"neutral"}\NormalTok{)}
\end{Highlighting}
\end{Shaded}

In the lp.control function, we specify sense=``max'' indicating that we
need to maximize the objective function. verbose = ``neutral'' is to
suppress any error report.

\textbf{Setting the objective function}

\begin{Shaded}
\begin{Highlighting}[]
\NormalTok{m2 <-}\StringTok{ }\KeywordTok{set.objfn}\NormalTok{(model,}\DataTypeTok{obj=}\KeywordTok{c}\NormalTok{(-}\DecValTok{2000}\NormalTok{,}\DecValTok{5000}\NormalTok{))}
\end{Highlighting}
\end{Shaded}

\textbf{Incorporating the constraints}

\begin{Shaded}
\begin{Highlighting}[]
\CommentTok{# Demand constraint}
\NormalTok{m3 <-}\StringTok{ }\KeywordTok{set.bounds}\NormalTok{(model, }\DataTypeTok{lower=}\KeywordTok{c}\NormalTok{(}\DecValTok{100}\NormalTok{,}\DecValTok{80}\NormalTok{))}

\CommentTok{#Production constraint}
\NormalTok{m4 <-}\StringTok{ }\KeywordTok{set.bounds}\NormalTok{(model, }\DataTypeTok{upper =} \KeywordTok{c}\NormalTok{(}\DecValTok{200}\NormalTok{,}\DecValTok{170}\NormalTok{))}
\end{Highlighting}
\end{Shaded}

The set.bounds is used to fix the lower and upper limits for the
variables in the model.

\textbf{The shipping constraint is added by the add.constraint function}

\begin{Shaded}
\begin{Highlighting}[]
\NormalTok{m5 <-}\StringTok{ }\KeywordTok{add.constraint}\NormalTok{(model, }\KeywordTok{c}\NormalTok{(}\DecValTok{1}\NormalTok{,}\DecValTok{1}\NormalTok{),}\StringTok{">="}\NormalTok{,}\DecValTok{200}\NormalTok{)}
\end{Highlighting}
\end{Shaded}

\textbf{Print Model}

To help generating the model with the constraints in a readable format
we add column and row names.

\begin{Shaded}
\begin{Highlighting}[]
\NormalTok{rownames <-}\StringTok{ }\KeywordTok{c}\NormalTok{(}\StringTok{"Shipping constraint"}\NormalTok{)}
\NormalTok{colnames <-}\StringTok{ }\KeywordTok{c}\NormalTok{(}\StringTok{"A"}\NormalTok{, }\StringTok{"B"}\NormalTok{)}
\KeywordTok{dimnames}\NormalTok{(model) <-}\StringTok{ }\KeywordTok{list}\NormalTok{(rownames, colnames)}
\KeywordTok{name.lp}\NormalTok{(model, }\StringTok{"Maximize profit"}\NormalTok{)}
\KeywordTok{print}\NormalTok{(model)}
\end{Highlighting}
\end{Shaded}

\begin{verbatim}
## Model name: Maximize profit
##                          A      B         
## Maximize             -2000   5000         
## Shipping constraint      1      1  >=  200
## Kind                   Std    Std         
## Type                  Real   Real         
## Upper                  200    170         
## Lower                  100     80
\end{verbatim}

\textbf{Solving the model}

The solve function help us the find the optimized solution for the
model.

\begin{Shaded}
\begin{Highlighting}[]
\KeywordTok{solve}\NormalTok{(model)}
\end{Highlighting}
\end{Shaded}

\begin{verbatim}
## [1] 0
\end{verbatim}

The output zero indicates that the model has been resolved and a
solution is generated.

** Optimum values of decision variables**

get. variables provide the most optimum values of A and B

\begin{Shaded}
\begin{Highlighting}[]
\KeywordTok{get.variables}\NormalTok{(model)}
\end{Highlighting}
\end{Shaded}

\begin{verbatim}
## [1] 100 170
\end{verbatim}

The optimum values of A and B are 100 and 170 respectively.

\textbf{Maximized Objective}

get. objective provides the maximized profit

\begin{Shaded}
\begin{Highlighting}[]
\KeywordTok{get.objective}\NormalTok{(model)}
\end{Highlighting}
\end{Shaded}

\begin{verbatim}
## [1] 650000
\end{verbatim}

The maximized profit is \$650000

\subsubsection{Application of Linear
Programming}\label{application-of-linear-programming}

\textbf{Shelf Space Optimization}

This is practical problem faced by super markets to decided which
product must be displayed in the shelves with maximum customer
visibility and very high probability of being purchased in the customer
even if it is not there in their shopping list.

However, the retailer needs to consider various factor before deciding
the products to be displayed

\begin{itemize}
\tightlist
\item
  No of products and the brands to be displayed
\item
  Profit margin of the product
\item
  Position of the product in the shelve
\item
  Promotional offer to be considered
\item
  Inventory cost
\item
  Demand
\item
  Expiration date. All the factor acts constraints for the LP problem
  with objective to maximize the revenue.
\end{itemize}

\textbf{Partner matching in Dating Sites}

Online dating sites use linear programming to match opposite sex
partners which satisfy the maximum number of preferences specified by
the members.

Each member has a preference score that is generated based the response
to the questionarie provided at the time of signing up.

An optimal dating equilibrium consist of a pairing of couples such that
there is no other allocation of females to males that are feasible. In
other words, the best allocation is one where people get the best
partner they can and are able to get.

Here the constraint for females is to be matched up with males whose
score is relatively larger than their own score. In simple words, it
means they have been able to get male who surpasses their expectation.
Whereas in case of male it is to minimize the score which would mean the
relative attractiveness of the female exceeds their expectation.
Promotion of ads on Television channels In this case the objective
function is maximize the audience viewership. On the other hands, the
constraints could be in the form of budget, maximum number of slot for
promotion, time slots etc.

Other areas are

\begin{itemize}
\item
  Airlines Revenue Management
\item
  Allocation of budget to an advertisement campaign on different medium.
\item
  Transportation problem
\item
  Call centre staffing and shift management.
\end{itemize}

\subsubsection{References}\label{references}

\url{https://www.analyticsvidhya.com/blog/2016/09/a-beginners-guide-to-shelf-space-optimization-using-linear-programming/}

\url{https://masterofeconomics.org/2009/08/15/simplex-algorithm-and-how-dating-web-sites-match-singles/}

\url{https://businessjargons.com/duality-in-linear-programming.html}

\url{https://businessjargons.com/linear-programming.html}

\url{https://www.youtube.com/watch?v=a2QgdDk4Xjw\&list=PLAD23E7AEFE221F70}

\url{https://www.r-bloggers.com/linear-programming-in-r-an-lpsolveapi-example/}

\url{https://www.r-bloggers.com/modeling-and-solving-linear-programming-with-r-free-book/}

\url{http://lpsolve.sourceforge.net/5.5/R.htm}


\end{document}
